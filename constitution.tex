\documentclass[11pt, oneside]{article}
\usepackage{geometry}
\usepackage{hyperref}
\geometry{a4paper}
\renewcommand{\thesection}{\arabic{section}} 
\renewcommand{\theenumi}{\alph{enumi}}
\renewcommand{\theenumii}{\roman{enumii}}
\begin{document}
\title{\textbf{Association rules for Greater Brisbane Inc.}}
\date{January 18, 2024}
\maketitle
\tableofcontents

\newpage

\section{Name}
\begin{enumerate}
\item The association will be known as Greater Brisbane Inc. trading as Greater Brisbane \textit{(Alternative if rejected: Greater Brisbane Urbanists Inc. trading as Greater Brisbane)}
\item The association’s registered Australian Business Number is \textit{[xx]}
\end{enumerate}

\section{Objects}
\begin{enumerate}

\item Greater Brisbane advocates for better urban fabric, from urban planning, housing and development, the building and construction industry, transport, tax and finance and any other issue that makes a city better, fairer and more sustainable. 
\item Greater Brisbane provides a voice for people typically excluded from public discourse about their city — renters, young families, aspirational residents, migrants and anyone who wants a growing, changing city.
\item Greater Brisbane seeks to build coalitions between decision-makers, academics, policy thinkers, not-for-profit organisations, industry and fellow urbanists to achieve housing abundance and a better city. 
\item Greater Brisbane undertakes research into best practice and reform pathways for Queensland to achieve our vision of better, fairer and more sustainable cities. 
\item Greater Brisbane may join, donate to or otherwise support any other not-for-profit organisation whose objects broadly align with these objects. Membership of other organisations will be displayed on the association website.
\item Greater Brisbane is a political non-partisan organisation open to anyone who aligns with our objects and rules. 
\item Greater Brisbane does not tolerate harassment, abuse, discrimination, racism, sexism, homophobia or transphobia. 
\end{enumerate}

\section{New membership}
\begin{enumerate}

\item An application must be made by the person applying for membership, in the form required by the management committee, and lodged with payment of a membership fee if required. 
\item Applications are referred to the management committee for approval or rejection. If an application is not rejected within six weeks of submission, it is deemed to be accepted. 
\item Members may exercise all rights of an ordinary member from the submission of their membership application until such time that their application is rejected by the management committee or their membership is otherwise terminated. 
\end{enumerate}

\section{Classes of membership}
\begin{enumerate}
\item Ordinary members are individuals who have duly applied to join, paid a membership fee if applicable, and are not disqualified based on the membership qualifications below. There is no limit on the number of ordinary members the association may have.
\item Organisation members are any not-for-profit organisation who align with the objects of the association, applied to join, paid a membership fee if applicable, and are accepted by a special resolution of a management committee meeting. Organisation members may not vote in general meetings, join the management committee or count for quorum at any meeting. There is no limit on the number of organisation members the association may have. 
\end{enumerate}
\section{Membership qualifications}
\begin{enumerate}
\item Members may not:
\begin{enumerate}
\item Own, be a board member or be in executive management of a for-profit enterprise in the property development, building construction, real estate or asset management sectors, or
\item Hold significant financial interests in or otherwise profit from real property, or
\item Bring the association into disrepute or otherwise be in breach of the objects of this association.
\end{enumerate}
\item Qualifications may be waived on a case-by-case basis by the explicit agreement of a general meeting of the membership. 
\end{enumerate}
\section{Membership fees}
\begin{enumerate}
\item The management committee may set membership fees from time to time at their discretion. 
\item The management committee may waive individuals’ or cohorts’ membership fee at their discretion.
\end{enumerate}

\section{Termination of membership}
\begin{enumerate}

\item Members may resign or terminate their membership by writing. Members who terminate their membership in this fashion are not entitled to refunds on their membership fees. 
\item Membership will be terminated if a member has been unfinancial for more than three months. 
\item The management committee may also agree by consensus to terminate a member’s membership if their actions or circumstances are in contravention of the conditions of membership. 
\item Members who die or, in the case of an organisation, wind up are deemed to have had their membership terminated at the time of their death or winding up. 
\end{enumerate}

\section{Register of members}
\begin{enumerate}
\item The management committee must keep a register of members which include their full name, email address and any other particulars the management committee decide. 
\item The register of members is not open for inspection to ordinary members and is only available for review or use by the management committee. 
\end{enumerate}

\section{Meetings}
\begin{enumerate}
\item All meetings may be held digitally through an electronic meeting method, via telephone, in person, or in a hybrid form of these methods.
\item Decisions at meetings will be made by consensus unless agreed by the meeting to hold a vote. In the event a vote is agreed to, a simple majority of members present will pass the motion. 
\item The appointment of proxies is not permitted, including for the purpose of voting or making quorum.
\item Minutes of general and management committee meetings are not required to be signed for accuracy or tabled at subsequent meetings, but must be made available for review for ordinary members on request. 
\end{enumerate}

\section{General meetings}
\begin{enumerate}
\item General meetings of the membership must be convened at least once per year, within six months of the end of the financial year. 
\item Quorum for a general meeting is the number of management committee members plus one, or the square root of the total membership rounded up to the next whole number, whichever is higher. 
\end{enumerate}
\section{Management committee}
\begin{enumerate}

\item The management committee consists of five ordinary members elected by ordinary members present at a general meeting to manage the day-to-day affairs of the association, make decisions on its behalf, act as spokespeople for the association and any other duties consistent with these rules. 
\item Management committee must meet at least four times a year and as frequently as necessary to undertake their work.
\item Quorum at a management committee meeting is a simple majority. 
\item A written resolution agreed to by three or more management committee members, for example a flying minute by email, is valid and effectual. 
\item The management committee may nominate a member of the management committee to act as chair, secretary or treasurer of the association at their discretion. 
\item A general meeting of the membership may decide to increase the number of members on the management committee or the roles the management committee may appoint from their number at their discretion. 
\item Any ordinary member is eligible to nominate to join the management committee. Nominations may be accepted until a ballot is conducted. 
\item Ballots may be conducted in writing or through an electronic voting platform using any proportional preferential voting method. 
\item If fewer nominations have been received than there are positions available on the management committee, those nominees are deemed to be elected without the need to conduct a ballot. 
\item If a member of the management committee resigns, is removed by consensus of the remaining committee members or does not attend more than three consecutive management committee meetings, that member is removed from the management committee and their casual vacancy may be filled by the management committee from any ordinary member of the association.
\end{enumerate}
\section{Donations}
\begin{enumerate}
\item The association may not accept donations from for-profit enterprises in the property development, building construction, real estate or asset management sectors, nor from any owner, board member or executive in any of these enterprises. 
\item Donations over \$500 will be disclosed in a public register on the website for at least three years from receipt of the donation. 
\end{enumerate}

\section{Bylaws}
\begin{enumerate}
\item The management committee or a general meeting may make, amend or rescind bylaws consistent with these rules.
\end{enumerate}

\section{Financial year}
\begin{enumerate}
\item The end date of the association’s financial year is 30 June 2024 in each year.
\end{enumerate}

\section{Rules otherwise}
\begin{enumerate}
\item But for these rules, \href{https://www.qld.gov.au/law/laws-regulated-industries-and-accountability/queensland-laws-and-regulations/associations-charities-and-non-for-profits/incorporated-associations/running-an-incorporated-association/rules-for-associations}{Queensland’s model rules for associations} as provided by section 47(1) of the \href{https://www.legislation.qld.gov.au/view/html/inforce/current/act-1981-074}{Associations Incorporation Act 1981} apply.
\end{enumerate}

\end{document}
